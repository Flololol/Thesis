%%%%%%%%%%%%%%%%%%%%%%%%%%%%%%%%%%%%%%%%%%%%%%%%%%%%%%%%%%%%%%%%%%%%%%%%
\chapter{Related Work}
%%%%%%%%%%%%%%%%%%%%%%%%%%%%%%%%%%%%%%%%%%%%%%%%%%%%%%%%%%%%%%%%%%%%%%%%

For a long time, not a lot of work has been published in the field of
uncertainty visualization, that tries to incorporate the level of error,
accuracy or confidence into the representation. This was until 2011 Hege
\etal{\cite{PMC}} introduced Probabilistic Marching Cubes, as a way to
model the level crossing probability of isovalues for a cell in an
uncertain scalar field. Their results were the probability for the
occurence of Marching Cube cases in random field realizations.\\
\indent In the following year, Hege \etal\ presented their approach for
probabilistic local features in uncertain vector fields~\cite{PLF},
extending the extraction of features from crisp vector fields to
uncertain fields using Monte-Carlo integration. Also in 2012, Holger
Theisel and Mathias Otto published an approach for the analysis of
vortex regions in uncertain vector fields, combining the Parallel
Vectors Operator~\cite{PV} with a Monte-Carlo sampling of a cell from
the uncertain fields together with its neighborhood. They obtained
volume renderings for the probabilities of the existence of a vortex
core line or region in a field.\\
\indent In this year, closely to the end of this work, Theisel \etal\
published the Approximate Parallel Vectors Operator~\cite{APV}. With
this, they expanded their method from 2012 to regions where the velocity
field was maximally parallel to its acceleration field, instead of
exactly parallel. This was necessary, as structures where all field are
parallel are unstable with uncertainty, hindering exact calculation. The
problem of certain uncertainty extraction will be a part of this work
too.\\