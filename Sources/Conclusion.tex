%%%%%%%%%%%%%%%%%%%%%%%%%%%%%%%%%%%%%%%%%%%%%%%%%%%%%%%%%%%%%%%%%%%%%%%%
\chapter{Conclusion}\label{chap:Discu}
%%%%%%%%%%%%%%%%%%%%%%%%%%%%%%%%%%%%%%%%%%%%%%%%%%%%%%%%%%%%%%%%%%%%%%%%

This thesis took the idea of uncertain vortex core line detection and
transferred it to ridges in uncertain scalar fields. We deeply discussed
the adjustments to the considered neighborhood to achieve Gaussian
distributed derivatives and how the different approaches for the
generation of samples from the uncertain fields influence the results.\\
\indent While ridge lines in three dimensions almost directly follow the
work of Otto and Theisel, ridges of co-dimension one needed some
adjustments. As the conservative approach to extract the uncertain
ridges with a modification of the Marching Cubes algorithm yielded
problems at the locations where the ridge feature was weak already, we
made use of the information that eigenvectors give us about their
gradients, to develop a criterion that estimates the existence of a
ridge, rather than strictly calculating it. Essentially, the criterion
approximates the transformation of the gradient along the direction
perpendicular to a ridge with the directional derivatives denoted by the
eigenvalues, to estimate if the gradient in some distance $d$ flipped
its direction and therefore passed a ridge. This gave really smooth
results without undesired features, but with the downside of losing
precision, especially for mostly certain sets. There are a lot of
different ways to obtain the ridges, but the optimal parameters for the
extraction depend on the uncertainty of the data set, making general
recommendations difficult.\\
\indent For the future, a point wise implementation of the new criterion
should definitely be considered, as the criterion is independent of
neighboring nodes. This would only be more congruent with the overall
mechanism of the criterion. Furthermore this would challenge the
aliasing problem, occuring if the ridge is not aligned with an axis of
the grid, as well as reducing the arithmetical effort. For the
3-dimensional case, this would decrease the size of the covariance
matrix from $80 \times 80$ to $25 \times 25$ dimensional, boosting the
computation time and challenging the issues with the numerical stability
of either decomposition. Also, a filter should be implemented, that
changes the type of extracted feature, depending on their relevance at
the location. This is possible based on the eigenvalues, even together
with a point wise extraction and would further improove the results.\\
\indent The methods described in this work were implemented for discrete
scalar fields of identical resolution as a plugin for ParaView. With
this plugin, we produced our results we used to evaluate and compare
the different approaches.
