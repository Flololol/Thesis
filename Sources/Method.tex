%%%%%%%%%%%%%%%%%%%%%%%%%%%%%%%%%%%%%%%%%%%%%%%%%%%%%%%%%%%%%%%%%%%%%%%%
\chapter{Method}\label{chap:Method}
%%%%%%%%%%%%%%%%%%%%%%%%%%%%%%%%%%%%%%%%%%%%%%%%%%%%%%%%%%%%%%%%%%%%%%%%

This chapter explains our principle approach on how to extract ridge
features from uncertain scalar fields. In the usual case, ridges do not
lie directly on the nodes of a cell, but inbetween. Therefore it is
insufficient to only look at a single node of every member of the set of
possibly correlated scalar fields to decide on the presence of a ridge.
Thus in the $3$-dimensional case the actual node at the location we
examine is the bottom left node of a cell of 8 nodes (see Figure~REF).
This follows an implementation detail, as the scalar fields are usually
traversed from bottom left to top right. Sampling these 8 nodes with a
Monte-Carlo method would only give us Gaussian distributed values over
the range of values at these locations, but no information about their
change in either dimension. Otto and Theisel~\cite{Vortex} showed in
their work that the derivative of Gaussian sampled cells without
considering their neighborhood is non-Gaussian distributed. Therefore we
add the 24 nodes around the 8 nodes of the cell to be able to calculate
the gradient via central differences. As we also need the Hessian for
every of the 8 nodes, we need the gradients of the 24 nodes surrounding
the cell. This leads to 80 nodes making up the cell we want to draw
samples from (see Figure~REF). With this we can create the uncertain
scalar field from Section~\ref{sec:USF} with the underlying mean vector
and covariance matrix fields.

%%%%%%%%%%%%%%%%%%%%%%%%%%%%%%%%%%%%%%%%%%%%%%%%%%%%%%%%%%%%%%%%%%%%%%%%
\section{Multivariate Gaussian Sampling}\label{sec:MGS}
%%%%%%%%%%%%%%%%%%%%%%%%%%%%%%%%%%%%%%%%%%%%%%%%%%%%%%%%%%%%%%%%%%%%%%%%

To analyse the uncertain scalar field obtained in Section~\ref{sec:USF},
we need to draw samples from the underlying multivariate gaussian
distribution.

%%%%%%%%%%%%%%%%%%%%%%%%%%%%%%%%%%%%%%%%%%%%%%%%%%%%%%%%%%%%%%%%%%%%%%%%
\section{Processing the cell}\label{sec:proc}
%%%%%%%%%%%%%%%%%%%%%%%%%%%%%%%%%%%%%%%%%%%%%%%%%%%%%%%%%%%%%%%%%%%%%%%%
