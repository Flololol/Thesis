%%%%%%%%%%%%%%%%%%%%%%%%%%%%%%%%%%%%%%%%%%%%%%%%%%%%%%%%%%%%%%%%%%%%%%%%
\chapter{Method}\label{chap:Method}
%%%%%%%%%%%%%%%%%%%%%%%%%%%%%%%%%%%%%%%%%%%%%%%%%%%%%%%%%%%%%%%%%%%%%%%%

This chapter explains our principle approach on how to extract ridge
features from uncertain scalar fields. In the usual case, ridges do not
lie directly on the nodes of a cell, but inbetween. Therefore it is
insufficient to only look at single nodes of every member of the set of
possibly correlated scalar fields to decide on the presence of a ridge.
Thus in the $3$-dimensional case the actual node at the location we
examine is the bottom left node of a cell of 8 nodes (see red node in
Figure~\ref{fig:3DNH}). This follows an implementation detail, as our
scalar fields are traversed from bottom left to top right. Sampling
these 8 nodes with a Monte-Carlo method would only give us Gaussian
distributed values over the range of values at these locations, but no
information about their change in either dimension outside the cell.
Otto and Theisel~\cite{Vortex} tested in their work to derive a velocity
vector $v$ in a two dimensional uncertain vector field by sampling the
vector and its four neighboring vectors individually from their
distributions. They then estimated the Jacobian $J$ with central
differences on the neighboring vectors and multiplied it with the
velocity vector, to get the acceleration vector $a = J v$. The resulting
vectors were not Gaussian distributed anymore. They solved this issue by
instead of sampling each vector individually, they sample the whole
neighborhood of the vector to keep track of its change along the
dimensions. Therefore we also inspect the 24 nodes adjacent to the 8
nodes of the cell, to be able to calculate the gradient via central
differences. As Theisel \etal{} were searching for vortex core lines in
uncertain vector fields, they only needed to derive their field once, to
be able to decide on the presence of a core line based on the
eigenspace. As we base our detections on eigenvalues and vectors as
well, we need to derive our uncertain scalar field twice. We therefore
also add the adjacent nodes of the 24 supporting nodes to our considered
neighborhood. This leads to 80 nodes making up the cell we want to draw
samples from (see Figure~\ref{fig:NH}). With this we can create the
uncertain scalar field from Section~\ref{sec:USF}. It is important to
remember that we cannot compute ridge lines and surfaces directly on the
uncertain scalar field, but on samples from it. Thus we sample the field
multiple times and the result is a probability for the existence of the
desired ridge feature in the cell.

%%%%%%%%%%%%%%%%%%%%%%%%%%%%%%%%%%%%%%%%%%%%%%%%%%%%%%%%%%%%%%%%%%%%%%%%
\section{Multivariate Gaussian Sampling}\label{sec:MGS}
%%%%%%%%%%%%%%%%%%%%%%%%%%%%%%%%%%%%%%%%%%%%%%%%%%%%%%%%%%%%%%%%%%%%%%%%

Our uncertain scalar field $S_{\mathcal{N}}$ at point
$x=(x_1,\dots,x_n)$ consists of a multivariate Gaussian distribution,
made up from the mean vector $\mu_x$ and the covariance matrix
$\Sigma_x$. The mean vector portrays the average values of the
neighborhood of $x$ while the covariance matrix contains the variances
of each value to each other around this mean. Unfortunately we cannot
compute ridge criteria from these objects. Monte-Carlo methods offer a
way to draw samples from multivariate normal distributions. The basis of
multivariate Gaussian sampling is the affine transformation property of
normal distributed vectors that states that any linear transformation of
a normal vector is again normal:

\begin{equation}
    X \sim \mathcal{N}(\mu,\Sigma) \Rightarrow AX \sim \mathcal{N}(A\mu, A\Sigma A)
\end{equation}

\noindent for any dimensionality $d$ of vector $X$ and $d \times d$
matrix $\Sigma$ and any $k \times d$ matrix $A$. By this property we
know that if we have a standard normal distributed vector $Z \sim
\mathcal{N}(0,I)$ with $I$ being an $d \times d$ identity matrix, then
$X = AZ + \mu$ is from the distribution $X \sim \mathcal{N}(\mu, A
A^\top)$. Now we simply have to find a matrix $A$ for which $AA^\top =
\Sigma$. The Cholesky (Section~\ref{sec:cholesky}) and the
Eigendecomposition (Section~\ref{sec:eigen}) are the two most popular
methods to calculate such a matrix as the Cholesky decomposition is
computationally efficient and the Eigendecomposition deliveres more
numerical stability as well as a lot of predefined librarys. As our
covariance matrices are only positive semi-definite, we need the
extended version of Cholesky. Now we get the two decompositions:
\begin{align}
    &\Sigma_x = L D L^\top \\
    &\Sigma_x = E \Lambda E^{-1},
\end{align}
In theory, a suitable matrix $A$ would now be either $A=L
D^{\frac{1}{2}}$ or $A=E\Lambda^{\frac{1}{2}}$ for $\Sigma = AA^\top$.
In practice, because of our relatively huge $80 \times 80$ dimensional
covariance matrix we get negative elements in the diagonal matrix $D$
even though our covariance matrix is positive semi-definite. Therefore
we cannot take the root of $D$. The same issue goes for the
Eigendecomposition where the eigenvalues become infinitesimally small
negative values, hindering the computation of the root of $\Lambda$. As
the sign of the eigenvalues determines the definiteness of a matrix,
this makes our covariance matrix technically indefinite, as we have
positive and negative eigenvalues. We solved this issue by simply
multiplying the components without taking the root of the diagonal
matrix, hence $A = LD$ and $A = E \Lambda$. This did not seem to affect
the samples drawn with the Eigendecomposition, but the samples generated
with Cholesky showed some inconsistencies. Section (INSERT REF) will
give more insight on that topic and compare the decompositions. Now we
can proceed with the found $A$. In the 3 (2) dimensional  case we
generate an 80 (24) dimensional standard uniformly distributed vector
and transform it into a standard normal distributed vector $Z$ using the
Box-Muller transformation (Section~\ref{sec:BM}). Now we multiply $A$
with $Z$ and add the mean vector of this location to get the sample
vector $X = AZ + \mu_x, X \sim \mathcal{N}(\mu_x, \Sigma_x)$. This is
only a single instance from the multivariate distribution, thus we
repeat the sampling multiple times per cell to try to model the
distribution as best as possible. With the now obtained sample vectors
we can start to calculate ridge criteria at the location.

%%%%%%%%%%%%%%%%%%%%%%%%%%%%%%%%%%%%%%%%%%%%%%%%%%%%%%%%%%%%%%%%%%%%%%%%
\section{Ridge Extraction}\label{sec:ridgeextract}
%%%%%%%%%%%%%%%%%%%%%%%%%%%%%%%%%%%%%%%%%%%%%%%%%%%%%%%%%%%%%%%%%%%%%%%%

In this section we will cover the extraction of a ridgesurface in a
3-dimensional scalar field. The 80-dimensional sample vector is split up
according to the original nodes and then the 32 gradients of the first
derivative are calculated via central differences. From them we estimate
the 8 Hessians of the cell gradients. The conservative approach to
extract the ridge surface would now be to calculate the dot product of
the gradient and the minor eigenvector at every node and apply the
modified Marching Cubes algorithm for iso value 0.
