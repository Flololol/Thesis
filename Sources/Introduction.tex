%%%%%%%%%%%%%%%%%%%%%%%%%%%%%%%%%%%%%%%%%%%%%%%%%%%%%%%%%%%%%%%%%%%%%%%%
\chapter{Introduction}
%%%%%%%%%%%%%%%%%%%%%%%%%%%%%%%%%%%%%%%%%%%%%%%%%%%%%%%%%%%%%%%%%%%%%%%%

For the analysis of scalar or vector fields, the extraction of features
enables the deeper understanding of the domains. With today's wide
range of possibilities of gaining data, feature extraction has become an
important field of research in scientific visualization. Using the
available computing power, simulations increasingly become the method of
choice to understand a problem or behaviour of a system. These simulations,
run with varying parameters to cover multiple scenarios, produce a lot
of data, that needs to be further processed. As the data usually is very
similar with only slight changes due to the parameters, individual
examination is inpractical and does not deliver an understanding of the
overall distribution of the data.\\
\indent Ridges are used in a variety of scenarios, like medical or flow
visualizations. They denote the points, where the scalar fields are
locally maximal and therefore play a vital role for the comprehension of
the data. With modern graphics cards, simulations of complex flow
systems can be run on affordable hardware, without the need for larger
computing servers, increasing the amount of data massively. This brings
up the need for a simultaneous analysis of the distribution of ridges in
an uncertain scalar field, obtained, for example, from the members of
an simulation ensemble.\\
\indent We will create multivariate Gaussian distributions from the
fields we examine and use Monte-Carlo methods to sample these
distributions. With the samples, we can compute ridge criteria for a
local area of the uncertain field. As the extraction of ridges in a
certain setting already yields some problems with false positives, we
use the information that eigenvectors give us about the underlying
system, to develop a new criterion for ridges of co-dimension one, that
estimates the existence of a ridge in a small distance, rather than
strictly calculating it.\\
\indent In this work, we will explain the problems occuring with
sampling multivariate Gaussian distributions, when using a strict
approach for the extraction of ridges. Further, we implemented our
method as a plugin for the data analysis and visualization tool
ParaView, together with the conservative approaches for obtaining ridges
in 2 and 3D. At the end, we will deeply discuss the differences coming
from the multitude of possibilities for the calculation of the ridge
features.
