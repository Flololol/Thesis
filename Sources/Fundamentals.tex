%%%%%%%%%%%%%%%%%%%%%%%%%%%%%%%%%%%%%%%%%%%%%%%%%%%%%%%%%%%%%%%%%%%%%%%%
\chapter{Fundamentals}
%%%%%%%%%%%%%%%%%%%%%%%%%%%%%%%%%%%%%%%%%%%%%%%%%%%%%%%%%%%%%%%%%%%%%%%%

This chapter provides a basic understanding of the underlying methods
and concepts, that were required for this work.

%%%%%%%%%%%%%%%%%%%%%%%%%%%%%%%%%%%%%%%%%%%%%%%%%%%%%%%%%%%%%%%%%%%%%%%%
\section{Grids}
%%%%%%%%%%%%%%%%%%%%%%%%%%%%%%%%%%%%%%%%%%%%%%%%%%%%%%%%%%%%%%%%%%%%%%%%

In computational science, discrete data domains are often represented by
grids. The data itself is mostly saved at specific points of the grid 
(nodes), or in regions (cells), enclosed by surrounding nodes and the
respective connections (edges) between them. In more rare cases the
values are saved in the edges or the faces of the cells. The connectivity
of the nodes is given by the topology of the grid and therefore the shape
of the cells. There are three types of grids: scattered data, which has
no topology, hence no edges connecting the nodes, structured grids, which
have an implicit topology following the ordering of the nodes, with a
fixed number of nodes per dimension, as well as fixed cell types, and
unstructured grids, which only have irregular topology with varying cell
types. For the latter the topology has to be stored explicitly. Structured
grids can further be distinguished into uniform, rectilinear and
curvilinear (irregular) structured grids. The nodes in uniform grids are
equidistant for every dimension, whereas rectilinear grids may have irregular
spacings along either axis and curvilinear grids may have irregular spacings
between each grid node. This work focuses on uniform structured grids as
it makes it easier to compare multiple grids at specific points in the
domain.\\
INSERT PICTURE TO DIFFERENT GRID TYPES

%%%%%%%%%%%%%%%%%%%%%%%%%%%%%%%%%%%%%%%%%%%%%%%%%%%%%%%%%%%%%%%%%%%%%%%%
\section{Scalar Fields}
%%%%%%%%%%%%%%%%%%%%%%%%%%%%%%%%%%%%%%%%%%%%%%%%%%%%%%%%%%%%%%%%%%%%%%%%

An $n$-dimensional field with a single scalar value at every point in
space is called a scalar field. A simple example for a scalar field is a
height map of some geographical terrain. It has two dimensions with a
height value at every point encoded with color. In this work we will use
the notation $S(x)$ for the scalar value at point $x = (x_1,\dots,x_n)$
in the scalar field $S$ with $S: \real^n \rightarrow \real$. While in
continuous scalar fields the values of every point are defined by a
function, discrete scalar fields only have values at specific points in
space. As real world data is often acquired by measuring certain
locations and usually does not follow any graspable function, we will
focus on discrete fields in this work.\\
INSERT PICTURE OF HEIGHT MAP

%%%%%%%%%%%%%%%%%%%%%%%%%%%%%%%%%%%%%%%%%%%%%%%%%%%%%%%%%%%%%%%%%%%%%%%%
\subsection{Uncertain Scalar Fields}
%%%%%%%%%%%%%%%%%%%%%%%%%%%%%%%%%%%%%%%%%%%%%%%%%%%%%%%%%%%%%%%%%%%%%%%%

Notation of uncertainty.

%%%%%%%%%%%%%%%%%%%%%%%%%%%%%%%%%%%%%%%%%%%%%%%%%%%%%%%%%%%%%%%%%%%%%%%%
\section{Vector Fields}
%%%%%%%%%%%%%%%%%%%%%%%%%%%%%%%%%%%%%%%%%%%%%%%%%%%%%%%%%%%%%%%%%%%%%%%%

An $n$-dimensional vector field $V$ associates an $m$-dimensional vector
with every point in space $V(x_1,\dots,x_n):\real^n \rightarrow \real^m$.
If the vector field is the result of deriving a scalar field, thus the 
vector field is the gradient of the scalar field, the vector field is
called conservative. Conservative vector fields have the property, that
the line integral along any path connecting two points is always equal.
Since we are extracting features from scalar fields in this work, our
vector fields will always be conservative and $n = m$.

%%%%%%%%%%%%%%%%%%%%%%%%%%%%%%%%%%%%%%%%%%%%%%%%%%%%%%%%%%%%%%%%%%%%%%%%
\section{Tensor Fields}
%%%%%%%%%%%%%%%%%%%%%%%%%%%%%%%%%%%%%%%%%%%%%%%%%%%%%%%%%%%%%%%%%%%%%%%%

An $n$-dimensional tensor field $H$ associates an $l \times m$-dimensional
tensor with every point in space $H(x_1,\dots,x_n): \real^n \rightarrow
\real^{l \times m}$. Tensor fields are a generalization of scalar and
vector fields as a tensor with $l = m = 1$ would represent a scalar, and
with $l > 1$ and $m = 1$ a vector. In our case, the tensor field is a
result of deriving a vector field, thus $n = l = m$.

%%%%%%%%%%%%%%%%%%%%%%%%%%%%%%%%%%%%%%%%%%%%%%%%%%%%%%%%%%%%%%%%%%%%%%%%
\section{Derivatives}
%%%%%%%%%%%%%%%%%%%%%%%%%%%%%%%%%%%%%%%%%%%%%%%%%%%%%%%%%%%%%%%%%%%%%%%%

Since we are dealing with discrete scalar fields, we have no function we
could derive to get the underlying gradient field. Instead we have to
approximate the derivatives with finite difference methods. There are
three forms which are commonly used:\\
\\
\begin{inparaenum}[(a)]
  \item Forward Differences
  \begin{equation}
    \nabla S(x_i) = \frac{S(x_{i+1}) - S(x_i)}{h}
  \end{equation}
  \item Backward Differences
  \begin{equation}
    \nabla S(x_i) = \frac{S(x_i) - S(x_{i-1})}{h}
  \end{equation}
  \item Central Differences
  \begin{equation}
    \nabla S(x_i) = \frac{S(x_{i+1}) - S(x_{i-1})}{2h}
  \end{equation}
\end{inparaenum}
where $S(x_i)$ denotes the $i$-th point in one dimension and $h$ the
distance between two neighboring points. Forward and backward differences
are used at the borders of the field, when no previous or succeeding
point is available. As we will explain in chapter \ref{chap:Method},
with our current implementation we process cells with a fixed size and
therefore do not consider border cases, thus this work focuses on central
differences.

%----------------------------------------------------------------------%
\subsection{First Derivative}
%----------------------------------------------------------------------%



%%%%%%%%%%%%%%%%%%%%%%%%%%%%%%%%%%%%%%%%%%%%%%%%%%%%%%%%%%%%%%%%%%%%%%%%
\section{Linear Interpolation}
%%%%%%%%%%%%%%%%%%%%%%%%%%%%%%%%%%%%%%%%%%%%%%%%%%%%%%%%%%%%%%%%%%%%%%%%

%%%%%%%%%%%%%%%%%%%%%%%%%%%%%%%%%%%%%%%%%%%%%%%%%%%%%%%%%%%%%%%%%%%%%%%%
\section{Principal Component Analysis}
%%%%%%%%%%%%%%%%%%%%%%%%%%%%%%%%%%%%%%%%%%%%%%%%%%%%%%%%%%%%%%%%%%%%%%%%

%%%%%%%%%%%%%%%%%%%%%%%%%%%%%%%%%%%%%%%%%%%%%%%%%%%%%%%%%%%%%%%%%%%%%%%%
\section{Matrix Decompositions}
%%%%%%%%%%%%%%%%%%%%%%%%%%%%%%%%%%%%%%%%%%%%%%%%%%%%%%%%%%%%%%%%%%%%%%%%

%%%%%%%%%%%%%%%%%%%%%%%%%%%%%%%%%%%%%%%%%%%%%%%%%%%%%%%%%%%%%%%%%%%%%%%%
\section{Multivariate Gaussian Sampling}
%%%%%%%%%%%%%%%%%%%%%%%%%%%%%%%%%%%%%%%%%%%%%%%%%%%%%%%%%%%%%%%%%%%%%%%%

%%%%%%%%%%%%%%%%%%%%%%%%%%%%%%%%%%%%%%%%%%%%%%%%%%%%%%%%%%%%%%%%%%%%%%%%
\section{Ridges}
%%%%%%%%%%%%%%%%%%%%%%%%%%%%%%%%%%%%%%%%%%%%%%%%%%%%%%%%%%%%%%%%%%%%%%%%

%%%%%%%%%%%%%%%%%%%%%%%%%%%%%%%%%%%%%%%%%%%%%%%%%%%%%%%%%%%%%%%%%%%%%%%%
\section{Marching Ridges}
%%%%%%%%%%%%%%%%%%%%%%%%%%%%%%%%%%%%%%%%%%%%%%%%%%%%%%%%%%%%%%%%%%%%%%%%

%%%%%%%%%%%%%%%%%%%%%%%%%%%%%%%%%%%%%%%%%%%%%%%%%%%%%%%%%%%%%%%%%%%%%%%%
\section{Parallel Vectors}
%%%%%%%%%%%%%%%%%%%%%%%%%%%%%%%%%%%%%%%%%%%%%%%%%%%%%%%%%%%%%%%%%%%%%%%%

%%%%%%%%%%%%%%%%%%%%%%%%%%%%%%%%%%%%%%%%%%%%%%%%%%%%%%%%%%%%%%%%%%%%%%%%
\section{How?}
%%%%%%%%%%%%%%%%%%%%%%%%%%%%%%%%%%%%%%%%%%%%%%%%%%%%%%%%%%%%%%%%%%%%%%%%

The package tries to be easy to use. If you are satisfied with the
default settings, just add
%
\begin{verbatim}
\documentclass{mimosis}
\end{verbatim}
%
at the beginning of your document. This is sufficient to use the class.
It is possible to build your document using either \LaTeX|, \XeLaTeX, or
\LuaLaTeX. I personally prefer one of the latter two because they make
it easier to select proper fonts.

%%%%%%%%%%%%%%%%%%%%%%%%%%%%%%%%%%%%%%%%%%%%%%%%%%%%%%%%%%%%%%%%%%%%%%%%
\section{Features}
%%%%%%%%%%%%%%%%%%%%%%%%%%%%%%%%%%%%%%%%%%%%%%%%%%%%%%%%%%%%%%%%%%%%%%%%

The template automatically imports numerous convenience packages that
aid in your typesetting process. lists the
most important ones. Let's briefly discuss some examples below. Please
refer to the source code for more demonstrations.

%%%%%%%%%%%%%%%%%%%%%%%%%%%%%%%%%%%%%%%%%%%%%%%%%%%%%%%%%%%%%%%%%%%%%%%%
\subsection{Typesetting mathematics}
%%%%%%%%%%%%%%%%%%%%%%%%%%%%%%%%%%%%%%%%%%%%%%%%%%%%%%%%%%%%%%%%%%%%%%%%

This template uses \verb|amsmath| and \verb|amssymb|, which are the
de-facto standard for typesetting mathematics. Use numbered equations
using the \verb|equation| environment.
%
If you want to show multiple equations and align them, use the
\verb|align| environment:
%
\begin{align}
    V &:= \{ 1, 2, \dots \}\\
    E &:= \big\{ \left(u,v\right) \mid \dist\left(p_u, p_v\right) \leq \epsilon \big\}
\end{align}
%
Define new mathematical operators using \verb|\DeclareMathOperator|.
Some operators are already pre-defined by the template, such as the
distance between two objects. Please see the template for some examples. 
%
Moreover, this template contains a correct differential operator. Use \verb|\diff| to typeset the differential of integrals:
%
\begin{equation}
  f(u) := \int_{v \in \domain}\dist(u,v)\diff{v}
\end{equation}
%
You can see that, as a courtesy towards most mathematicians, this
template gives you the possibility to refer to the real numbers~$\real$
and the domain~$\domain$ of some function. Take a look at the source for
more examples. By the way, the template comes with spacing fixes for the
automated placement of brackets.

%%%%%%%%%%%%%%%%%%%%%%%%%%%%%%%%%%%%%%%%%%%%%%%%%%%%%%%%%%%%%%%%%%%%%%%%
\subsection{Typesetting text}
%%%%%%%%%%%%%%%%%%%%%%%%%%%%%%%%%%%%%%%%%%%%%%%%%%%%%%%%%%%%%%%%%%%%%%%%

Along with the standard environments, this template offers
\verb|paralist| for lists within paragraphs.
%
Here's a quick example: The American constitution speaks, among others, of
%
\begin{inparaenum}[(a)]
  \item life
  \begin{equation}
    V = 5
  \end{equation}
  \item liberty
  \item the pursuit of happiness.
\end{inparaenum}
%
These should be added in equal measure to your own conduct. To typeset
units correctly, use the \verb|siunitx| package. For example, you might
want to restrict your daily intake of liberty to \SI{750}{\milli\gram}.

Likewise, as a small pet peeve of mine, I offer specific operators for \emph{ordinals}. Use \verb|\th| to typeset things like July~4\th correctly. Or, if you are referring to the 2\nd edition of a book, please use \verb|\nd|. Likewise, if you came in 3\rd in a marathon, use \verb|\rd|. This is my 1\st rule.

%%%%%%%%%%%%%%%%%%%%%%%%%%%%%%%%%%%%%%%%%%%%%%%%%%%%%%%%%%%%%%%%%%%%%%%%
\section{Changing things}
%%%%%%%%%%%%%%%%%%%%%%%%%%%%%%%%%%%%%%%%%%%%%%%%%%%%%%%%%%%%%%%%%%%%%%%%

Since this class heavily relies on the \verb|scrbook| class, you can use
\emph{their} styling commands in order to change the look of things. For
example, if you want to change the text in sections to \textbf{bold} you
can just use
%
\begin{verbatim}
  \setkomafont{sectioning}{\normalfont\bfseries}
\end{verbatim}
%
at the end of the document preamble---you don't have to modify the class
file for this. Please consult the source code for more information.
