%%%%%%%%%%%%%%%%%%%%%%%%%%%%%%%%%%%%%%%%%%%%%%%%%%%%%%%%%%%%%%%%%%%%%%%%
% Some adjustments to make the bibliography more clean
%%%%%%%%%%%%%%%%%%%%%%%%%%%%%%%%%%%%%%%%%%%%%%%%%%%%%%%%%%%%%%%%%%%%%%%%
%
% The subsequent commands do the following:
%  - Removing the month field from the bibliography
%  - Fixing the Oxford commma
%  - Suppress the "in" for journal articles
%  - Remove the parentheses of the year in an article
%  - Delimit volume and issue of an article by a colon ":" instead of
%    a dot ""
%  - Use commas to separate the location of publishers from their name
%  - Remove the abbreviation for technical reports
%  - Display the label of bibliographic entries without brackets in the
%    bibliography
%  - Ensure that DOIs are followed by a non-breakable space
%  - Use hair spaces between initials of authors
%  - Make the font size of citations smaller
%  - Fixing ordinal numbers (1st, 2nd, 3rd, and so) on by using
%    superscripts

% Remove the month field from the bibliography. It does not serve a good
% purpose, I guess. And often, it cannot be used because the journals
% have some crazy issue policies.
\AtEveryBibitem{\clearfield{month}}
\AtEveryCitekey{\clearfield{month}}

% Fixing the Oxford comma. Not sure whether this is the proper solution.
% More information is available under [1] and [2].
%
% [1] http://tex.stackexchange.com/questions/97712/biblatex-apa-style-is-missing-a-comma-in-the-references-why
% [2] http://tex.stackexchange.com/questions/44048/use-et-al-in-biblatex-custom-style
%
\AtBeginBibliography{%
  \renewcommand*{\finalnamedelim}{%
    \ifthenelse{\value{listcount} > 2}{%
      \addcomma
      \addspace
      \bibstring{and}%
    }{%
      \addspace
      \bibstring{and}%
    }
  }
}

% Suppress "in" for journal articles. This is unnecessary in my opinion
% because the journal title is typeset in italics anyway.
\renewbibmacro{in:}{%
  \ifentrytype{article}
  {%
  }%
  % else
  {%
    \printtext{\bibstring{in}\intitlepunct}%
  }%
}

% Remove the parentheses for the year in an article. This removes a lot
% of undesired parentheses in the bibliography, thereby improving the
% readability. Moreover, it makes the look of the bibliography more
% consistent.
\renewbibmacro*{issue+date}{%
  \setunit{\addcomma\space}
    \iffieldundef{issue}
      {\usebibmacro{date}}
      {\printfield{issue}%
       \setunit*{\addspace}%
       \usebibmacro{date}}%
  \newunit}

% Delimit the volume and the number of an article by a colon instead of
% by a dot, which I consider to be more readable.
\renewbibmacro*{volume+number+eid}{%
  \printfield{volume}%
  \setunit*{\addcolon}%
  \printfield{number}%
  \setunit{\addcomma\space}%
  \printfield{eid}%
}

% Do not use a colon for the publisher location. Instead, connect
% publisher, location, and date via commas.
\renewbibmacro*{publisher+location+date}{%
  \printlist{publisher}%
  \setunit*{\addcomma\space}%
  \printlist{location}%
  \setunit*{\addcomma\space}%
  \usebibmacro{date}%
  \newunit%
}

% Ditto for other entry types.
\renewbibmacro*{organization+location+date}{%
  \printlist{location}%
  \setunit*{\addcomma\space}%
  \printlist{organization}%
  \setunit*{\addcomma\space}%
  \usebibmacro{date}%
  \newunit%
}

% Do not abbreviate "technical report".
\DefineBibliographyStrings{english}{%
  techreport = {technical report},
}

% Display the label of a bibliographic entry in bare style, without any
% brackets. I like this more than the default.
%
% Note that this is *really* the proper and official way of doing this.
\DeclareFieldFormat{labelnumberwidth}{#1\adddot}

% Ensure that DOIs are followed by a non-breakable space.
\DeclareFieldFormat{doi}{%
  \mkbibacro{DOI}\addcolon\addnbspace
    \ifhyperref
      {\href{http://dx.doi.org/#1}{\nolinkurl{#1}}}
      %
      {\nolinkurl{#1}}
}

% Use proper hair spaces between initials as suggested by Bringhurst and
% others.
\renewcommand*\bibinitdelim {\addnbthinspace}
\renewcommand*\bibnamedelima{\addnbthinspace}
\renewcommand*\bibnamedelimb{\addnbthinspace}
\renewcommand*\bibnamedelimi{\addnbthinspace}

% Make the font size of citations smaller. Depending on your selected
% font, you might not need this.
\renewcommand*{\citesetup}{%
  \biburlsetup
  \small
}

\DeclareLanguageMapping{british}{bibliography-correct-ordinals}
\DeclareLanguageMapping{english}{bibliography-correct-ordinals}
